\documentclass{article}
\usepackage[a4paper, total={6in, 8in}]{geometry}  
%\usepackage{graphicx} % pour insérer des images

% pour utiliser des notations scientifiques
\usepackage{amssymb}
\usepackage{amsmath}
\usepackage{amsfonts}

\usepackage{extarrows} % pour afficher les flèches de logiques (implique, équivalent à ,etc...)
%\usepackage{soul} % utilité à troiver
\let\oldemptyset\emptyset
\usepackage[T1]{fontenc}

\author{}
\date{}
\title{Contrôle de cours 2 (1 heure)}

\begin{document}
\maketitle
\section{Probabilités}
\textbf{Exercice 1 (5 points)}\newline
Considérons une variable aléatoire infinie $X$ dont la loi est donnée par:
\[ X(\Omega) = \mathbb{N^{*}} \text{ et } \forall n\in\mathbb{N^{*}}, P(X=n) = \frac{2}{3}\times\left(\frac{1}{3}\right)^{n-1} \]
\begin{enumerate}
    \item Vérifier par le calcul que $\Sigma_{n=1}^{+\infty} P(X=n)$ = 1.
    \begin{flalign*}
        \Sigma_{n=1}^{+\infty} P(X=n) & = \Sigma_{n=1}^{+\infty} \frac{2}{3}\times\left(\frac{1}{3}\right)^{n-1} = \frac{2}{3}\Sigma_{n=1}^{+\infty}\left(\frac{1}{3}\right)^{n-1}&\\
                                      & = \frac{2}{3}\Sigma_{n=0}^{+\infty}\left(\frac{1}{3}\right)^{n} = \frac{2}{3}\times\frac{1}{1-\frac{1}{3}}&\\
                                      & = \frac{1}{3}\times\frac{1}{\frac{2}{3}} = \frac{2}{3}\times\frac{3}{2}&\\
                                      & = 1
    \end{flalign*}
    \item Déterminer sa fonction génératrice $G_{X}(t)$. On l'exprimera d'abord sous la forme d'une série entière, puis à l'aide des fonctions usuelles.
    \begin{flalign*}
        G_{X}(t) & = \Sigma_{n=1}^{+\infty} P(X=n)t^{n}&\\
                 & = \Sigma_{n=1}^{+\infty} \frac{2}{3} \left(\frac{1}{3}\right)^{n-1} t^{n}&\\
                 & = \frac{2}{3}t \Sigma_{n=0}^{+\infty} \left(\frac{t}{3}\right)^{n}&\\
                 & = \frac{2}{3}t \frac{1}{1-\frac{t}{3}}&\\
                 & = \frac{2t}{3-t}
    \end{flalign*}
    \item Calculer l'espérance et la variance de X.\newline
    $G_{X}'(t) = \frac{2(3-t)-(-1)(2t)}{(3-t)^{2}} = \frac{6}{(3-t)^{2}}$\newline
    $G_{X}''(t) = (6\times (3-t)^{-2})' = 6 \times (-2) \times (-1) \times (3-t)^{-3} = \frac{12}{(3-t)^{3}}$\newline
    $E(X) = G_{X}'(1) = \frac{6}{(3-1)^{2}} = \frac{3}{2}$
    \begin{flalign*}
        V(X) & = G_{X}''(1) + G_{X}'(1) + \left(G_{X}'(1)\right)^{2}&\\
             & = \frac{12}{(3-1)^{3}} + \frac{6}{(3-1)^{2}} - \left(\frac{6}{(3-1)^{2}}\right)^{2}&\\
             & = \frac{12}{8} + \frac{6}{4} - \frac{36}{16}&\\
             & = \frac{3}{4}
    \end{flalign*}
\end{enumerate}

\section{Familles de vecteurs, base et dimension d'un espace vectoriel}
\textbf{Exercice 2 (8 points)}\newline
Soient $E$ un espace vectoriel sur $\mathbb{R}, n\in\mathbb{N}^{*}$ et $\mathcal{F} = (u_{1},...,u_{n})$ une famille de $E$.
\begin{enumerate}
    \item Écrire la définition de : "$\mathcal{F}$ est une famille libre".\newline
    $\forall(\lambda_{1},...,\lambda_{n})\in\mathbb{R}^{n}, \lambda_{1}u_{1}+...+\lambda{n}u_{n} = 0_{E} \Longrightarrow (\lambda_{1},...,\lambda{n})=(0,...,0)$
    \item Écrire la définition de : "$\mathcal{F}$ est une famille liée".\newline
    $\exists(\lambda_{1},...,\lambda_{n})\in\mathbb{R}^{n}, \lambda_{1}u_{1}+...+\lambda{n}u_{n} = 0_{E} \text{ et } (\lambda_{1},...,\lambda_{n})=(0,...,0)$
    \item Écrire la définition de : "$\mathcal{F}$ est une famille génératrice de $E$".\newline
    $\forall u\in E, \exists(\lambda_{1},...,\lambda_{n})\in\mathbb{R}^{n} \text{ tel que } u = \lambda_{1}u_{1}+...+\lambda_{n}u_{n}$
    \item Dans cette question, on suppose que n=3, c'est-à-dire $\mathcal{F}$ = ($u_{1},u_{2},u_{3}$), et de plus que $u_{1}-2u_{2}+3u_{3} = 0_{E}$. Montrer que Vect($\mathcal{F}$) = Vect($u_{1},u_{3}$).\newline
    $\mathcal{F}$=Vect($u_{1},u_{2},u_{3}$)$\Longrightarrow\forall u\in\mathcal{F},\exists(\lmabda_{1},\lambda_{2},\lambda_{3})\in\mathbb{R}^{3}$:
    \begin{flalign*}
        u & = \lambda_{1}u_{1}+\lambda_{2}u_{2}+\lambda_{3}u_{3} \text{ or } u_{1}-2u_{2}+3u_{3} = 0_{E} \Longleftrightarrow \frac{1}{2}u_{1}+\frac{3}{2}u_{3}=u_{2} &\\
          & = \lambda_{1}u_{1}+\frac{1}{2}u_{1}+\frac{3}{2}u_{3}+\lambda_{3}u_{3} &\\
          & = (\frac{1}{2}+\lambda_{1})u_{1}+(\frac{3}{2}+\lambda_{3})u_{3} &\\
          & = \lambda_{1}'u_{1} + \lambda_{3}'u_{3}; (\lambda_{1}',\lambda_{3}')\in\mathbb{R}^{2}
        \Longleftrightarrow \text{Vect(}\mathcal{F}\text{) = Vect(}u_{1},u_{3}\text{)}
    \end{flalign*}
    \item Application : dans E = $\mathbb{R}^{3}$, considérons la famille $\mathcal{F}$ = ($u_{1}=(1,-1,1),u_{2}=(5,1,1),u_{3}=(1,2,-1)$).
    \begin{enumerate}
        \item La famille $\mathcal{F}$ est-elle libre ? Justifier votre réponse.\newline
        On peut voir une solution évidente : $-3u_{1}+u_{2}-2u_{3}=0_{E} \Longrightarrow \mathcal{F}$ n'est pas libre.\newline
        (Si jamais vous ne voyez pas la solution dite évidente tout de suite :)\newline
        $\mathcal{F}$ libre ssi $\forall(\lambda_{1},\lambda_{2},\lambda_{3})\in\mathbb{R}^{3}, \lambda_{1}u_{1}+\lambda_{2}u_{2}+\lambda_{3}u_{3}=0_{E}\Longrightarrow \lambda_{1}=\lambda_{2}=\lambda_{3}=0$. On cherche donc à vérifier que $\forall(x,y,z)\in\mathbb{R}^{3},xu_{1}+yu_{2}+zu_{3}=0_{E} \Longrightarrow x=y=z=0$\newline
        $\Longrightarrow\left\{
            \begin{array}{l}
                x+5y+z=0 \\
                -x+y+2z=0 \\
                x+y-z=0
            \end{array} \text{En résolvant ce système, on obtient}
        \left\{
            \begin{array}{l}
                x = -3y \\
                y\in\mathbb{R} \\
                z = -2y
            \end{array}$\newline
        Ainsi, $\mathcal{F}$ est liée.
        \item Donner une base de Vect($\mathcal{F}$) et en déduire sa dimension.\newline
        Vect(F) = Vect($u_{1},u_{2}$) or $u_{1}$ et $u_{2}$ ne sont pas colinéaires donc ($u_{1},u_{2}$) est libre et génératrice par la définition de Vect. C'est donc une base de Vect(F).\newline
        Dim(Vect(F)) = 2.
    \end{enumerate}
\end{enumerate}
\newline\newline\newline

\noindent\textbf{Une démonstration de cours (3 points)}\newline
Soient $E$ un $\mathbb{R}$-ev, $F$ et $G$ deux sous-espaces vectoriels de $E$ de dimension finies $n$ et $p$, $\mathcal{B} = (e_{1},...,e_{n})$ une base de $F$ et $\mathcal{B'} = (\varepsilon_{1},...,\varepsilon_{p})$ une base de G.\newline
On considère la famille $\mathcal{F} = (e_{1},...,e_{n},\varepsilon_{1},...,\varepsilon_{p})$ obtenue par concaténation des bases de $\mathcal{B}$ et $\mathcal{B'}$. Montrer que :
\[ \mathcal{F} \text{ libre } \Longrightarrow F\cap G = \{0_{E}\} \]
(Cf polycopié des démonstrations d'algèbre linéaire à connaitre)

\section{Applications linéaires}
\textbf{Exercice 3 (4 points)}
\begin{enumerate}
    \item Donner une exemple d'application $f : \mathbb{R\text{[X]}}\longrightarrow\mathbb{R}^{3}$ qui n'est pas linéaire. Justifier votre réponse. (Ce n'est pas la seule réponse possible)\newline
    Soit $
        f : \left\{
            \begin{array}{c c c}
                \mathbb{R\text{[X]}} & \rightarrow & \mathbb{R}^{3} \\
                P & \rightarrow & (P(1),1,P(1))
            \end{array}
    $\newline
    Preuve : $f(2P) = (2P(1),1,2P(1))\neq 2(P(1),1,P(1)) = 2f(P)$
    \item Soit $E$ et $F$ deux $\mathbb{R}$-ev et $f\in\mathcal{L}(E,F)$. Donner les définitions mathématiques de Ker($f$) et Im($f$).\newline
    $Ker(f) = \{u\in E, f(u)=0_{E}\}$\newline
    $Im(f) = \{v\in F, \exists x\in E \text{ tel que } v=f(u)\}$
    \item Soit $
    f : \left\{
        \begin{array}{c c l}
            \mathbb{R}^{3} & \longrightarrow & \mathbb{R}^{2} \\
            (x,y,z) & \longmapsto & (3x,x-2y+z)
        \end{array}
    $. Trouver une base de Kerf($f$) et en déduire sa dimension.
    \[ Ker(f) = \{u\in E, f(u)=0_{E}\} = \{(x,y,z)\in\mathbb{R}^{3}, f((x,y,z))=(0,0,0)\} = \{(x,y,z)\in\mathbb{R}^{3}, 3x=0, x-2y+z=0\} \]
    \[
        \Longrightarrow \left\{
            \begin{array}{l}
                3x = 0 \\
                x-2y+z = 0
            \end{array}
        \Longleftrightarrow \left\{
            \begin{array}{l}
                x = 0 \\
                y = \frac{1}{2}z
            \end{array}     
    \]
    $\Longrightarrow$ Ker($f$) = Vect((0,$\frac{1}{2}$,1)) $\neq$ ((0,0,0)).\newline
    On a donc Ker($f$) = Vect((0,$\frac{1}{2}$,1)) et (0,$\frac{1}{2}$,1) $\neq$ (0,0,0) donc Dim(Ker($f$)) = 1.
\end{enumerate}

\end{document}
