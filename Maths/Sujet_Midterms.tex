\documentclass{article}
% pour modifier les marges
% les nombres ne correspondent pas à la taille des marges (à gauche et à droite) mais à la place dans laquelle on peut écrire à gauche et à droite.
\usepackage[a4paper, total={6in, 8in}]{geometry}    

%\usepackage{graphicx} % pour insérer des images

% pour utiliser des notations scientifiques
\usepackage{amssymb}
\usepackage{amsmath}
\usepackage{amsfonts}

\usepackage{extarrows} % pour afficher les flèches de logiques (implique, équivalent à ,etc...)
%\usepackage{soul} % utilité à trouver
%\let\oldemptyset\emptyset
%\usepackage[T1]{fontenc}


\author{}
\date{}
\title{}

\begin{document}
\maketitle
\begin{centering}
    \item \noindent\textbf{\Huge EPITA}\newline\newline
    \item \noindent\textbf{\huge Mathématiques}\newline\newline
    \item \noindent\textbf{Contrôle de mi-semestre S3}\newline\newline
    \item \noindent\textbf{Octobre 2023}\newline\newline
    \item \noindent\textbf{Durée : 3 heures}\newline\newline
    \item 
\end{centering}

\noindent\textbf{Nom: }\newline\newline
\noindent\textbf{Prénom: }\newline\newline
\noindent\textbf{Classe: }\newline\newline\newline\newline

\noindent\textbf{\Large NOTE: }\newline
Le barème est sur 40 points. La note sera ramenée sur 20 par une simple division par 2.\newline

\noindent\rule{\textwidth}{1pt}
\textbf{\Large Consigne:}
\begin{itemize}
    \item[-] Lire l'énoncé entier avant de commencer. Il y en a en tout 7 exercices.
    \item[-] Si vous parvenez pas à démontrer un résultat donné explicitement dans l'énoncé d'une question, vous pouvez admettre ce résultat et continuer l'exercice.
    \item[-] Documents et calculatrices interdits.
    \item[-] Répondre directement sur les feuilles jointes, dans les espaces prévus. Aucune autre feuille ne sera corrigée.
    \item[-] Ne pas écrire au crayon de papier.
\end{itemize}
\noindent\rule{\textwidth}{1pt}


\newpage
\noindent\textbf{Exercice 1 (6 points)}
\begin{footnotesize}
\begin{enumerate}
    \item Déterminer la nature de la série de terme général: $u_{n} = \ln(\cos(\frac{1}{n}))$. Justifier proprement.\newline\newline
    ...........................................................................................................................................................................\newline
    ...........................................................................................................................................................................\newline
    ...........................................................................................................................................................................\newline
    ...........................................................................................................................................................................\newline
    ...........................................................................................................................................................................\newline
    ...........................................................................................................................................................................\newline
    ...........................................................................................................................................................................\newline
    ...........................................................................................................................................................................\newline
    ...........................................................................................................................................................................\newline
    ...........................................................................................................................................................................\newline\newline
    \item Déterminer la nature de la série de terme général: $u_{n} = \frac{(n!)^{2}}{(3n)!}$. Justifier proprement.\newline\newline
    ...........................................................................................................................................................................\newline
    ...........................................................................................................................................................................\newline
    ...........................................................................................................................................................................\newline
    ...........................................................................................................................................................................\newline
    ...........................................................................................................................................................................\newline
    ...........................................................................................................................................................................\newline
    ...........................................................................................................................................................................\newline
    ...........................................................................................................................................................................\newline
    ...........................................................................................................................................................................\newline
    ...........................................................................................................................................................................\newline
    ...........................................................................................................................................................................\newline
    ...........................................................................................................................................................................\newline
    ...........................................................................................................................................................................\newline
    ...........................................................................................................................................................................\newline
    ...........................................................................................................................................................................\newline\newline
    \item Déterminer la nature de la série de terme général: $u_{n} = \frac{(-1)^{n}}{n\ln(n)}$. Justifier proprement.\newline\newline
    ...........................................................................................................................................................................\newline
    ...........................................................................................................................................................................\newline
    ...........................................................................................................................................................................\newline
    ...........................................................................................................................................................................\newline
    ...........................................................................................................................................................................\newline
    ...........................................................................................................................................................................\newline
    ...........................................................................................................................................................................\newline
    ...........................................................................................................................................................................\newline
    ...........................................................................................................................................................................\newline
    ...........................................................................................................................................................................\newline
    ...........................................................................................................................................................................\newline
    ...........................................................................................................................................................................\newline
    ...........................................................................................................................................................................\newline
    ...........................................................................................................................................................................\newline
    ...........................................................................................................................................................................\newline\newline
\end{enumerate}\newpage
\end{footnotesize}
\noindent\textbf{Exercice 2 (6 points)}\newline
\begin{footnotesize}
Considérons la série de terme général $u_{n} = \frac{(-1)^{n}}{\sqrt{n}+(-1)^{n}}$.
\begin{enumerate}
    \item Trouver a $\in \mathbb{R}$ tel que $u_{n} = \frac{(-1)^{n}}{\sqrt{n}} + \frac{a}{n} + \circ(\frac{1}{n})$.\newline\newline
    ...........................................................................................................................................................................\newline
    ...........................................................................................................................................................................\newline
    ...........................................................................................................................................................................\newline
    ...........................................................................................................................................................................\newline
    ...........................................................................................................................................................................\newline
    ...........................................................................................................................................................................\newline
    ...........................................................................................................................................................................\newline
    ...........................................................................................................................................................................\newline
    ...........................................................................................................................................................................\newline
    ...........................................................................................................................................................................\newline\newline
    \item Déterminer la nature de $\Sigma u_{n}$.\newline\newline
    ...........................................................................................................................................................................\newline
    ...........................................................................................................................................................................\newline
    ...........................................................................................................................................................................\newline
    ...........................................................................................................................................................................\newline
    ...........................................................................................................................................................................\newline
    ...........................................................................................................................................................................\newline
    ...........................................................................................................................................................................\newline
    ...........................................................................................................................................................................\newline
    ...........................................................................................................................................................................\newline
    ...........................................................................................................................................................................\newline
    ...........................................................................................................................................................................\newline
    ...........................................................................................................................................................................\newline
    ...........................................................................................................................................................................\newline
    ...........................................................................................................................................................................\newline
    ...........................................................................................................................................................................\newline\newline
    \item Montrer que $u_{n}\sim \frac{(-1)^{n}}{\sqrt{n}}$.\newline\newline
    ...........................................................................................................................................................................\newline
    ...........................................................................................................................................................................\newline
    ...........................................................................................................................................................................\newline
    ...........................................................................................................................................................................\newline
    ...........................................................................................................................................................................\newline
    ...........................................................................................................................................................................\newline
    ...........................................................................................................................................................................\newline
    ...........................................................................................................................................................................\newline\newline
    \item Les séries $\Sigma u_{n}$ et $\Sigma \frac{(-1)^{n}}{\sqrt{n}}$ sont-elles de même nature? Expliquer.\newline\newline
    ...........................................................................................................................................................................\newline
    ...........................................................................................................................................................................\newline\newline
\end{enumerate}\newpage
\end{footnotesize}
\noindent\textbf{Exercice 3 (7 points)}\newline
\begin{footnotesize}
Soit a$\in$]0,$\pi$[. On considère la suite ($u_{n}$) définie pour tout n$\in \mathbb{N^{*}}$ par:
\[ u_{n} = n! \times \prod_{k=1}^{n} \sin\left(\frac{a}{k}\right) = n! \times \left(\sin\left(\frac{a}{1}\right)\sin\left(\frac{a}{2}\right)...\sin\left(\frac{a}{n}\right)\right) \]

\noindent On admet que cette suite ($u_{n}$) est strictement positive. Le but de l'exercice est d'étudier la nature de $\Sigma u_{n}$ en fonction de a.
\begin{enumerate}
    \item On suppose dans cette question que a$\neq$1. En utilisant la règle de d'Alembert, discuter la nature de $\Sigma u_{n}$ en fonction de a.\newline\newline
    ...........................................................................................................................................................................\newline
    ...........................................................................................................................................................................\newline
    ...........................................................................................................................................................................\newline
    ...........................................................................................................................................................................\newline
    ...........................................................................................................................................................................\newline
    ...........................................................................................................................................................................\newline
    ...........................................................................................................................................................................\newline\newline
    \item On suppose dans cette question que a=1. Considérons la série $\Sigma\ln(n\sin(\frac{1}{n}))$ et la suite ($S_{n}$) de ses sommes partielles.
    \begin{enumerate}
        \item Montrer que pour tout n$\in \mathbb{N^{*}}$, $S_{n}=\ln(u_{n})$.\newline\newline
        ...........................................................................................................................................................................\newline
        ...........................................................................................................................................................................\newline
        ...........................................................................................................................................................................\newline
        ...........................................................................................................................................................................\newline
        ...........................................................................................................................................................................\newline\newline
        \item Étudier la nature de $\Sigma\ln(n\sin(\frac{1}{n}))$.\newline\newline
        ...........................................................................................................................................................................\newline
        ...........................................................................................................................................................................\newline
        ...........................................................................................................................................................................\newline
        ...........................................................................................................................................................................\newline
        ...........................................................................................................................................................................\newline
        ...........................................................................................................................................................................\newline
        ...........................................................................................................................................................................\newline
        ...........................................................................................................................................................................\newline
        ...........................................................................................................................................................................\newline\newline
        \item Que peut-on en déduire sur la suite ($u_{n}$)?\newline\newline
        ...........................................................................................................................................................................\newline
        ...........................................................................................................................................................................\newline
        ...........................................................................................................................................................................\newline
        ...........................................................................................................................................................................\newline\newline
        \item La série $\Sigma u_{n}$ est-elle convergente?\newline\newline
        ...........................................................................................................................................................................\newline
        ...........................................................................................................................................................................\newline
        ...........................................................................................................................................................................\newline
        ...........................................................................................................................................................................
    \end{enumerate}
\end{enumerate}
\end{footnotesize}
\newpage
\noindent\textbf{Exercice 4 (5,5 points)}\newline
\begin{footnotesize}
Soient ($u_{n}$) et ($v_{n}$) deux suites réelles strictement positives.
\begin{enumerate}
    \item On suppose dans cette question que ($u_{n}$)$\leqslant$($v_{n}$) à partir d'un certain rang. Ainsi, il existe $n_{0} \in \mathbb{N}$ tel que
    \[ \forall n\in \mathbb{N}, n\geqslant n_{0} \Longrightarrow u_{n}\leqslant v_{n}\]
    Dans chacune des expressions ci-dessous, remplacer les pointillés par un des symboles $\Longrightarrow$, $\Longleftarrow$ ou $\Longleftrightarrow$:
    \begin{enumerate}
        \item $\Sigma u_{n}$ converge ...... $\Sigma v_{n}$ converge.
        \item $\Sigma u_{n}$ diverge ...... $\Sigma v_{n}$ diverge.
    \end{enumerate}
    \item On suppose maintenant qu'au voisinage de +$\infty$, $u_{n}\sim v_{n}$.
    \begin{enumerate}
        \item Que peut-on dire des séries $\Sigma u_{n}$ et $\Sigma v_{n}$?\newline\newline
        ...........................................................................................................................................................................\newline
        ...........................................................................................................................................................................\newline\newline
        \item Démontrer cette propriéte. On pourra admettre sans démonstration les résultats de la question 1.\newline\newline
        ...........................................................................................................................................................................\newline
        ...........................................................................................................................................................................\newline
        ...........................................................................................................................................................................\newline
        ...........................................................................................................................................................................\newline
        ...........................................................................................................................................................................\newline
        ...........................................................................................................................................................................\newline
        ...........................................................................................................................................................................\newline
        ...........................................................................................................................................................................\newline
        ...........................................................................................................................................................................\newline
        ...........................................................................................................................................................................\newline
        ...........................................................................................................................................................................\newline
        ...........................................................................................................................................................................\newline
        ...........................................................................................................................................................................\newline
        ...........................................................................................................................................................................\newline
        ...........................................................................................................................................................................\newline
        ...........................................................................................................................................................................\newline
        ...........................................................................................................................................................................\newline
        ...........................................................................................................................................................................\newline
        ...........................................................................................................................................................................\newline
        ...........................................................................................................................................................................\newline
        ...........................................................................................................................................................................\newline
        ...........................................................................................................................................................................\newline
        ...........................................................................................................................................................................\newline
        ...........................................................................................................................................................................\newline
        ...........................................................................................................................................................................\newline
        ...........................................................................................................................................................................\newline
        ...........................................................................................................................................................................\newline
        ...........................................................................................................................................................................\newline
        ...........................................................................................................................................................................\newline
        ...........................................................................................................................................................................\newline
        ...........................................................................................................................................................................\newline
        ...........................................................................................................................................................................\newline
        ...........................................................................................................................................................................\newline
        ...........................................................................................................................................................................\newline
        ...........................................................................................................................................................................\newline
        ...........................................................................................................................................................................\newline
        ...........................................................................................................................................................................\newline
        ...........................................................................................................................................................................\newline
        ...........................................................................................................................................................................\newline
        ...........................................................................................................................................................................
    \end{enumerate}
\end{enumerate}\newpage
\end{footnotesize}
\noindent\textbf{Exercice 5 (6,5 points)}\newline
\begin{footnotesize}
Un étudiant passe un examen sous forme de QCM. L'examen contient 20 questions et chaque question est notée sur 1 point. La note totale de l'épreuve est donc une note sur 20. C'est un QCM sans points négatifs ni points intermédiaires: à chaque question, la note obtenue ne peut être que 0 ou 1.\newline
L'étudiant s'est mal préparé à l'examen et choisit de répondre au hasard. Ses réponses aux questions sont indépendantes et, pour chaque question, il a une même probabilités p $\in$]0,1[ que sa réponse soit juste.
\begin{enumerate}
    \item Pour tout k$\in$\{1,2,...,20\}, on définit la variable aléatoire $X_{k}$ = "Note de l'étudiant à la question k".
    \begin{enumerate}
        \item Soit k$\in$\{1,2,...,20\}. Donner la loi de $X_{k}$.\newline\newline
        ...........................................................................................................................................................................\newline
        ...........................................................................................................................................................................\newline
        ...........................................................................................................................................................................\newline\newline
        \item En déduire la fonction génératrice $G_{X_{k}}$ de $X_{k}$.\newline\newline
        ...........................................................................................................................................................................\newline\newline
        \item En utilisant $G_{X_{k}}$, calculer l'espérance et la variance de $X_{k}$.\newline\newline
        ...........................................................................................................................................................................\newline
        ...........................................................................................................................................................................\newline
        ...........................................................................................................................................................................\newline
        ...........................................................................................................................................................................\newline
        ...........................................................................................................................................................................\newline\newline
    \end{enumerate}
    \item Considérons la variable aléatoire Y = "Note totale obtenue par l'étudiant à l'épreuve."
    \begin{enumerate}
        \item Donner en justifiant la fonction génératrice de Y.\newline\newline
        ...........................................................................................................................................................................\newline
        ...........................................................................................................................................................................\newline
        ...........................................................................................................................................................................\newline\newline
        \item En déduire la loi de Y.\newline\newline
        ...........................................................................................................................................................................\newline
        ...........................................................................................................................................................................\newline
        ...........................................................................................................................................................................\newline
        ...........................................................................................................................................................................\newline
        ...........................................................................................................................................................................\newline\newline
        \item Calculer l'espérance et la variance de Y.\newline\newline
        ...........................................................................................................................................................................\newline
        ...........................................................................................................................................................................\newline
        ...........................................................................................................................................................................\newline
        ...........................................................................................................................................................................\newline
        ...........................................................................................................................................................................\newline
        ...........................................................................................................................................................................\newline
        ...........................................................................................................................................................................\newline
        ...........................................................................................................................................................................\newline
        ...........................................................................................................................................................................\newline
        ...........................................................................................................................................................................\newline
        ...........................................................................................................................................................................\newline
        ...........................................................................................................................................................................\newline
        ...........................................................................................................................................................................\newline
        ...........................................................................................................................................................................\newline
        ...........................................................................................................................................................................\newline\newline\newline
    \end{enumerate}
\end{enumerate}
\end{footnotesize}
\noindent\textbf{Exercice 6 (6 points)}\newline
\begin{footnotesize}
\begin{enumerate}
    \item Trouver le rayon de convergence $R_{1}$ de la série entière $\Sigma\frac{x^{n}}{n!}$. Justifier votre réponse.\newline\newline
    ...........................................................................................................................................................................\newline
    ...........................................................................................................................................................................\newline
    ...........................................................................................................................................................................\newline
    ...........................................................................................................................................................................\newline
    ...........................................................................................................................................................................\newline\newline
    \item Rappeler (sans justifier) une expression simple (à l'aide des fonctions usuelles) de sa fonction somme, définie pour tout x$\in$]-$R_{1}$;$R_{1}$[ par
    \[ f(x) = \Sigma_{n=0}^{+\infty} \frac{x^{n}}{n!} \]\newline
    ...........................................................................................................................................................................\newline\newline

    \item En déduire le rayon de convergence et une expression simple de la fonction somme de $\Sigma \frac{2^{n}}{n!}x^{n}$.\newline\newline
        ...........................................................................................................................................................................\newline
        ...........................................................................................................................................................................\newline
        ...........................................................................................................................................................................\newline\newline
    \item Trouver une expression simple de $\Sigma_{n=3}^{+\infty} \frac{x^{n}}{(n-3)!}$.\newline\newline
    ...........................................................................................................................................................................\newline
    ...........................................................................................................................................................................\newline
    ...........................................................................................................................................................................\newline\newline
    \item Démontrer que la fonction g : x $\mapsto \frac{1}{1+2x}$ peut se mettre sous la forme g(x) = $\Sigma_{n=0}^{+\infty} (-2)^{n} x^{n}$.\newline
    Quel est le rayon de convergence $R_{2}$ de cette série entière?\newline\newline
    ...........................................................................................................................................................................\newline
    ...........................................................................................................................................................................\newline
    ...........................................................................................................................................................................\newline
    ...........................................................................................................................................................................\newline
    ...........................................................................................................................................................................\newline\newline
    \item Exprimer sous la forme d'une série entière la fonction x $\mapsto \ln$(1+2x) et donner son rayon de convergence.\newline\newline
    ...........................................................................................................................................................................\newline
    ...........................................................................................................................................................................\newline
    ...........................................................................................................................................................................\newline
    ...........................................................................................................................................................................\newline
    ...........................................................................................................................................................................\newline\newline
    \item Exprimer sous la forme d'une série entière la fonction x $\mapsto \frac{x^{2}}{(1+2x)^{2}}$ et donner son rayon de convergence.\newline\newline
    ...........................................................................................................................................................................\newline
    ...........................................................................................................................................................................\newline
    ...........................................................................................................................................................................\newline
    ...........................................................................................................................................................................\newline
    ...........................................................................................................................................................................\newline
    ...........................................................................................................................................................................\newline
    ...........................................................................................................................................................................\newline
    ...........................................................................................................................................................................\newline\newline
\end{enumerate}
\end{footnotesize}
\noindent\textbf{Exercice 7 (4 points)}\newline
\begin{footnotesize}
Considérons une variable aléatoire entière X admettant une fonction génératrice de la forme $G_{X}(t) = ae^{2t}$ où a $\in \mathbb{R}$.
\begin{enumerate}
    \item Quelle est la valeur de a?\newline\newline
    ...........................................................................................................................................................................\newline
    ...........................................................................................................................................................................\newline\newline
    \item En écrivant $G_{X}(t)$ sous forme d'une série entière, en déduire la loi de X.\newline\newline
    ...........................................................................................................................................................................\newline
    ...........................................................................................................................................................................\newline
    ...........................................................................................................................................................................\newline
    ...........................................................................................................................................................................\newline\newline
    \item Calculer l'espérance et la variance de X.\newline\newline
    ...........................................................................................................................................................................\newline
    ...........................................................................................................................................................................\newline
    ...........................................................................................................................................................................\newline
    ...........................................................................................................................................................................\newline
    ...........................................................................................................................................................................\newline
    ...........................................................................................................................................................................\newline
    ...........................................................................................................................................................................\newline
    ...........................................................................................................................................................................\newline
    ...........................................................................................................................................................................\newline
    ...........................................................................................................................................................................\newline
    ...........................................................................................................................................................................\newline
    ...........................................................................................................................................................................
\end{enumerate}
\end{footnotesize}
\end{document}
