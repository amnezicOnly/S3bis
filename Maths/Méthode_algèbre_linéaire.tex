\documentclass{article}
\usepackage[a4paper, total={6in, 8in}]{geometry}  
%\usepackage{graphicx} % pour insérer des images

% pour utiliser des notations scientifiques
\usepackage{amssymb}
\usepackage{amsmath}
\usepackage{amsfonts}

\usepackage{extarrows} % pour afficher les flèches de logiques (implique, équivalent à ,etc...)
%\usepackage{soul} % utilité à troiver
\let\oldemptyset\emptyset
\usepackage[T1]{fontenc}

\author{}
\date{}
\title{Méthode Algèbre linéaire}

\begin{document}
\maketitle
\section{Base ou non?}
Soit F=\{$e_{1},...,e_{n}$\} une famille d'un K-EV E. Pour déterminier si F est une base de E:
\begin{enumerate}
    \item Si Dim(F)$\neq$Dim(E), alors F n'est pas une base de E
    \item Si Dim(F)=Dim(E) il faut vérifier si la famille F et libre ou génératrice.
\end{enumerate}
\section{Transformer une famille pour la rendre base}
Soit F=\{$E_{1},...,e_{p}$\} une famille d'un K-EV E de dimension n (n$\neq$p). Dim(F)$\neq$Dim(E) donc F n'est pas une base de E. On doit donc supprimer ou ajouter des vecteurs à la famille pour la transformer en base.\newline
\subsection{Suppression de vecteurs (p>n)}
On pose l'ensemble ($a_{1},...,a_{p}$)$\in \mathbb{R^{p}}$ tel que $a_{1}e_{1}+...+a_{p}e_{p} = O_{\mathbb{R^{p}}}$. Le but de la résolution de ce système permet de savoir quel vecteur \textbf{ne pas} enlever
\subsection{Ajout de vecteurs}

\section{Matrice de passage}
Exemple de matrice
$$\left(
    \begin{array}{cc}
    a & b \\
    c & d 
    \end{array} 
  \right)$$
\section{Déterminer la matrice d'une application linéaire}
\section{Trouver le noyau et l'image d'une matrice d'une application linéaire}
\section{Calculer le déterminant d'une matrice}
\section{Les projecteurs}
\end{document}
