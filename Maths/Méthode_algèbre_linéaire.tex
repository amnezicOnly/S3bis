\documentclass{article}
\usepackage[a4paper, total={6in, 8in}]{geometry}  
%\usepackage{graphicx} % pour insérer des images

% pour utiliser des notations scientifiques
\usepackage{amssymb}
\usepackage{amsmath}
\usepackage{amsfonts}

\usepackage{extarrows} % pour afficher les flèches de logiques (implique, équivalent à ,etc...)
%\usepackage{soul} % utilité à troiver
\let\oldemptyset\emptyset
\usepackage[T1]{fontenc}

\author{}
\date{}
\title{Méthode Algèbre linéaire}

\begin{document}
\maketitle
\section{Base ou non?}
Soit F=\{$e_{1},...,e_{n}$\} une famille d'un K-EV E. Pour déterminier si F est une base de E:
\begin{enumerate}
    \item Si Dim(F)$\neq$Dim(E), alors F n'est pas une base de E
    \item Si Dim(F)=Dim(E) il faut vérifier si la famille F et libre ou génératrice.
\end{enumerate}
\section{Transformer une famille pour la rendre base}
Soit F=\{$E_{1},...,e_{p}$\} une famille d'un K-EV E de dimension n (n$\neq$p). Dim(F)$\neq$Dim(E) donc F n'est pas une base de E. On doit donc supprimer ou ajouter des vecteurs à la famille pour la transformer en base.\newline
\subsection{Suppression de vecteurs (p>n)}
On pose l'ensemble ($a_{1},...,a_{p}$)$\in \mathbb{R^{p}}$ tel que $a_{1}e_{1}+...+a_{p}e_{p} = 0_{\mathbb{R^{p}}}$. Le but de la résolution de ce système permet de savoir quel vecteur \textbf{ne pas} enlever.
\subsection{Ajout de vecteurs}

\section{Matrice de passage}
Exemple de matrice
$$\left(
    \begin{array}{cc}
    a & b \\
    c & d 
    \end{array} 
  \right)$$
\section{Déterminer la matrice d'une application linéaire}
\section{Trouver le noyau et l'image d'une matrice d'une application linéaire}
\section{Calculer le déterminant d'une matrice carré}
Soit une A une matrice carrée de taille n $\mathcal{M_{n}}$, avec n$\in\mathbb{N\{0,1\}}$. Il y a deux situations possibles :
\begin{itemize}
  \item n = 2 : on a dans ce A = $\left(
  \begin{array}{c c}
    a & b \\
    c & d
  \end{array}
  \right)
  $, det(A) = ad-bc

  \item n>2 : on a donc la matrice suivante :
  \[
    A = \left(
      \begin{array}{c c c c}
        a_{1,1} & a_{1,2} & ... & a_{1,n} \\
        \vdots & \ddots & \ddots & \vdots \\
        a_{n,1} & a_{n,2} & ... & a_{n,n}
      \end{array}
    \right)  
  \]

  On va procéder par étape successives. Pour calculer le déterminant de A, on va développer par rapport à une ligne ou une colonne. On notera $\Delta_{ij}$ la matrice A sans la ligne i ni la colonne j. Si on développe selon la i-ième ligne ou colonne, on aura :
  \[ det(A) = \Sigma_{j=0}^{n} (-1)^{i+j} a_{i,j} det(\Delta_{i,j}) \]
  Pour calculer det($\Delta_{i,j}$), il suffira de réitérer l'étape précédente en posant $\Delta_{i,j}$ = A' par exemple.
\end{itemize}
\noindent Pour calculer un déterminant, on pourra se servir des propriétés suivantes : (on pose A,B$\in\mathcal{M_{n}}(\mathbb{K})$ et $\lambda\in\mathbb{K}$)
\begin{itemize}
  \item det($\lambda$A) = $\lambda^{n}$det(A)
  \item det(AB) = det(A)det(B)
  \item det()
  \item Si une ligne/colonne est une combinaison linéaire d'autres lignes/colonnes de la matrice, alors son déterminant est nul
  \item Si une ligne/colonne est composée uniquement de 0, son déterminant est nul
  \item Si on permute 2 lignes/colonnes, le déterminant est multiplié par -1
  \item Si on multiplie une ligne/colonne par un scalaire $\lambda$, alors son déterminant est aussi multiplié par $\lambda$
\end{itemize}
\section{Trouver les valeurs et vecteurs propres d'une matrice}
Tout d'abord, soient A une matrice carré de taille n, un vecteur colonne v de taille n et un scalaire $\lambda$, on dit que v/$\lambda$ est un vecteur/valeur propre de A ssi :
\[ Av = \lambda v \]

\subsection{Déterminer les vecteurs propres à partir d'une valeur propre $\lambda$:}
L'ensemble des vecteurs propres de A par la valeur propre $\lambda$ est :
\[ E_{\lambda}=\{v\in\mathbb{K^{n}} \| Av = \lambda v\} = Ker(A-\lambda I_{n})\]
On cherche donc l'ensemble des vecteurs v tel que :
\[\left(
  \begin{array}{c c c}
    a_{1,1}-\lambda & ... & a_{1,n} \\
    \vdots & a_{i,j}- \lambda & \vdots \\
    a_{n,1} & ... & a_{n,n} - \lambda
  \end{array}\right)\times
\left(
  \begin{array}{c}
    x_{1} \\
    \vdots \\
    x_{n}
  \end{array}
\right) = \left(
  \begin{array}{c}
    0 \\
    \vdots \\
    0
  \end{array}
\right)
\]
Ce qui va nous amener à résoudre le système suivant :
\[\left\{
  \begin{array}{l}
    a_{1,1}x_{1} + ... + a_{1,n}x_{n} = 0 \\
    \vdots \\
    a_{n,1}x_{1} + ... + a_{n,n}x_{n} = 0
  \end{array}

\]
\section{Les projecteurs}
\end{document}
