\documentclass{article}
\usepackage[a4paper, total={6in, 8in}]{geometry}  
%\usepackage{graphicx} % pour insérer des images

% pour utiliser des notations scientifiques
\usepackage{amssymb}
\usepackage{amsmath}
\usepackage{amsfonts}

\usepackage{extarrows} % pour afficher les flèches de logiques (implique, équivalent à ,etc...)
%\usepackage{soul} % utilité à troiver
\let\oldemptyset\emptyset
\usepackage[T1]{fontenc}

\author{}
\date{}
\title{Contrôle de cours 2 (1 heure)}

\begin{document}
\maketitle
\section{Probabilités}
\textbf{Exercice 1 (5 points)}\newline
Considérons une variable aléatoire infinie $X$ dont la loi est donnée par:
\[ X(\Omega) = \mathbb{N^{*}} \text{ et } \forall n\in\mathbb{N^{*}}, P(X=n) = \frac{2}{3}\times\left(\frac{1}{3}\right)^{n-1} \]
\begin{enumerate}
    \item Vérifier par le calcul que $\Sigma_{n=1}^{+\infty} P(X=n)$ = 1.
    \item Déterminer sa fonction génératrice $G_{X}(t)$. On l'exprimera d'abord sous la forme d'une série entière, puis à l'aide des fonctions usuelles.
    \item Calculer l'espérance et la variance de X.
\end{enumerate}

\section{Familles de vecteurs, base et dimension d'un espace vectoriel}
\textbf{Exercice 2 (8 points)}\newline
Soient $E$ un espace vectoriel sur $\mathbb{R}, n\in\mathbb{N}^{*}$ et $\mathcal{F} = (u_{1},...,u_{n})$ une famille de $E$.
\begin{enumerate}
    \item Écrire la définition de : "$\mathcal{F}$ est une famille libre".
    \item Écrire la définition de : "$\mathcal{F}$ est une famille liée".
    \item Écrire la définition de : "$\mathcal{F}$ est une famille génératrice de $E$".
    \item Dans cette question, on suppose que n=3, c'est-à-dire $\mathcal{F}$ = ($u_{1},u_{2},u_{3}$), et de plus que $u_{1}-2u_{2}+3u_{3} = 0_{E}$.\newline
    Montrer que Vect($\mathcal{F}$) = Vect($u_{1},u_{3}$).
    \item Application : dans E = $\mathbb{R}^{3}$, considérons la famille $\mathcal{F}$ = ($u_{1}=(1,-1,1),u_{2}=(5,1,1),u_{3}=(1,2,-1)$).
    \begin{enumerate}
        \item La famille $\mathcal{F}$ est-elle libre ? Justifier votre réponse.
        \item Donner une base de Vect($\mathcal{F}$) et en déduire sa dimension.
    \end{enumerate}
\end{enumerate}

\noindent\textbf{Une démonstration de cours (3 points)}\newline
Soient $E$ un $\mathbb{R}$-ev, $F$ et $G$ deux sous-espaces vectoriels de $E$ de dimension finies $n$ et $p$, $\mathcal{B} = (e_{1},...,e_{n})$ une base de $F$ et $\mathcal{B'} = (\varepsilon_{1},...,\varepsilon_{p})$ une base de G.\newline
On considère la famille $\mathcal{F} = (e_{1},...,e_{n},\varepsilon_{1},...,\varepsilon_{p})$ obtenue par concaténation des bases de $\mathcal{B}$ et $\mathcal{B'}$. Montrer que :
\[ \mathcal{F} \text{ libre } \Longrightarrow F\cap G = \{0_{E}\} \]

\section{Applications linéaires}
\textbf{Exercice 3 (4 points)}
\begin{enumerate}
    \item Donner une exemple d'application $f : \mathbb{R\text{[X]}}\longrightarrow\mathbb{R}^{3}$ qui n'est pas linéaire. Justifier votre réponse.
    \item Soit $E$ et $F$ deux $\mathbb{R}$-ev et $f\in\mathcal{L}(E,F)$. Donner les définitions mathématiques de Ker($f$) et Im($f$).
    \item Soit $
    f : \left\{
        \begin{array}{c c l}
            \mathbb{R}^{3} & \longrightarrow & \mathbb{R}^{2} \\
            (x,y,z) & \longmapsto & (3x,x-2y+z)
        \end{array}
    $. Trouver une base de Kerf($f$) et en déduire sa dimension.
\end{enumerate}

\end{document}
