\documentclass{article}
\usepackage{graphicx} % Required for inserting images
\usepackage{amssymb}
\usepackage{amsmath}
\usepackage{amsfonts}
\usepackage{extarrows}
\usepackage{soul}
\tolerance=1
\emergencystretch=\maxdimen
\hyphenpenalty=10000
\hbadness=10000
\let\oldemptyset\emptyset
\usepackage[T1]{fontenc}

\author{Amnézic}
\date{}
\title{Probabilités}

\begin{document}
\maketitle
\newpage
\tableofcontents
\newpage

\section{Définition}
\textbf{Définition}
\begin{quote}
    Soit $X$ une variable aléatoire finie entière. On note $X(\Omega)$ = [0,n] l'ensemble de ses valeurs possibles. Sa fonction génératrice est alors la fonction G$_{X}$ définie pour tout $t \in R$ par :
    \[ G_{X}(t) = E(t^{X}) = \sum_{k=0}^{n} P(X=k)t^{k} \]
\end{quote}

\noindent \textbf{Remarques}
\begin{enumerate}
    \item La fonction G$_{X}$ est un polynôme en t : \[ G_{X}(t) = \sum_{k=0}^{n} P(X=k)t^{k} = p_{0} + p_{1}t + ... + p_{n}t^{n} \]
    \item Si $\forall$k $\in$ [O,n], P(X=k)=P(Y=k) $\Longrightarrow$ X = Y
\end{enumerate}

\section{Calculs de l'espérance et de la variance}
\textbf{Théorème}
\begin{quote}
    Soient X une variable aléatoire finie entière et G$_{X}$ sa fonction génératrice. On a alors :
    \begin{itemize}
        \item G$_{X}$(1) = 1
        \item E($X$) = G$_{X}$'(1)
        \item Var($X$) = G$_{X}$''(1) + G$_{X}$'(1) - (G$_{X}$'(1))$^{2}$
    \end{itemize}
\end{quote}

La fonction G$_{X}$ contient toute l'information donnée par la loi de $X$. En particulier, on peut en déduire l'espérance et la variance de $X$.

\section{Somme de varaibles aléatoire}
\textbf{Théorème}
Soient 2 variables aléatoires finies entières X et Y \textbf{indépendantes}, de fonctions génératrices G$_{X}$(t) et G$_{Y}$(t). La somme $X + Y$ admet alors une fonction génératrice :
\[ G_{X+Y}(t) = G_{X}(t) \times G_{Y}(t) \]

(loi binomiale à revoir)

\section{Séries entières}
\subsection{Définition}
Soit x $\in R$. Une série entière est une série $\Sigma a_{n} x^{n}$ où $(a_{n})_{n \in N}$ est une suite réelle.Le terme générale de la série est donc une fonction de x de la forme $u_{n}$(x) = $a_{n}$(x)

\textbf{Remarques}
\begin{enumerate}
    \item La série est toujours convergente en x=0. En effet, pour cette valeur de x, les termes $u_{n}$(x)=$a_{n}x^{n}$ sont tous nuls sauf $u_{0}$(x)=$a_{0}$ donc la série converge vers $a_{0}$.
    \item Si la série converge en d'autres valeurs de x, alors sa limite dépend de x et on peut définir f(x) = $\sigma_{n=0}^{+\infty}a_{n}x^{n}$: l'ensemble des valeurs de x pour lesquelles la série converge est le domaine de convergence de la série. C'est aussi le domaine de définition $D_{f}$ de la fonction f.
\end{enumerate}

\subsection{Rayon de convergence}
\textbf{Définiton}
\begin{quote}
    Soient ($a_{n}$) une suite réelle, $\Sigma a_{n}x^{n}$ la série entière définie par cette suite et la fonction f(x) = $\sigma_{n=0}^{+\infty}a_{n}x^{n}$.\newline
    Alors il existe $R \in R_{+} \cup {+\infty}$ tel que :
    \begin{enumerate}
        \item $\forall x \in R tel que |x| ???$, la série converge absolument
        \item $\forall x \in R tel que |x| > R$, la série diverge
    \end{enumerate}
\end{quote}
\noindent R est appelé "rayon de convergence" de la série entière et ???.

\noindent \textbf{Interpétation}
\begin{enumerate}
    \item Si R = 0, la série converge vers $a_{0}$ en x = 0, et diverge en toute valeur x telle que |x|>0. Ainsi, le domaine de définition de f est $D_{f}$ = {0}.
    \item Si R$\in R_{+}^{*}$, 
\end{enumerate}


\end{document}