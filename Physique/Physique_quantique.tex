\documentclass{article}
%\usepackage{graphicx} % Required for inserting images
%\usepackage{amssymb}
%\usepackage{amsmath}
%\usepackage{amsfonts}
%\usepackage{extarrows}
%\usepackage{soul}
\tolerance=1
\emergencystretch=\maxdimen
\hyphenpenalty=10000
\hbadness=10000
\let\oldemptyset\emptyset
\usepackage[T1]{fontenc}

\author{}
\date{}
\title{Physique quantique}

\begin{document}
\maketitle

La mécanique quantique, est un domaine de la physique apparu au XXe siècle, qui explique le comportement au niveau atomique et subatomique de la matière associée avec l'énergie, que la physique dite classique ne peut expliquer.

\section{Différence entre mécanique traditionnelle et quantique:}
\begin{itemize}
    \item{traditionnelle}
        \begin{itemize}
        \item à l'échelle macroscopique
        \item énergie continue
        \item position déterminée avec une bonne précision
        \end{itemize}
    \item{quantique}
        \begin{itemize}
        \item à l'échelle nanoscopique
        \item énergie discontinue
        \item 1 quanta = 1 paquet d'énergie de valeur $h\nu$ = E (Loi de Planck); E = $\frac{hc}{\lambda}$ avec h constante de Planck en J.s-1, c la célérité de la lumière dans le vide et $\nu = \frac{c}{\lambda}$ en s-1 (c'est une fréquence)
        \item notion de probabilité : la position de l'élément étudié est incertaine
        \item phénomènes interprétés grâce à la mécanique quantique : analyse spectrale qui permet d'identifier la matière en ayant analysé le spectre émis par celle-ci :
            \begin{itemize}
            \item photo-électrique (création d'un courant suite à l'interaction entre le photon et l'électron (libre) du métal)
            \item corps noir (TD1)
            \item modèle de Bohr (TD2)
            \end{itemize}
        \end{itemize}
\end{itemize}
\section{Corps noir}
Un corps noir est un corps qui absorbe tous les rayonnements qu'on lui envoie et les réémet quand sa température augmente.
Les rayonnements coulissent des UV(faible longueur d'onde) aux IR(grandes longueurs d'ondes)

\section{Catastrophe ultraviolette}
Proposition de Rayleigh-Jeans: E est continue (<E> = $k_{B}$T avec $k_{B}$ constante) $\Longrightarrow$ u($\lambda$,T) diverge qd $\lambda\rightarrow$0 (autrement dit, se dirige vers les UV) $\Longrightarrow$ Catastrophe ultraviolette

\section{Proposition de Max Planck}
\begin{itemize}
    \item Énergie discrète (E = n$E_{0}$) et non continue
    \item <E> = $\frac{\Sigma_{n=1}^{+\infty} E \times e^{-\frac{E}{k_{B}T}}}{\Sigma_{n=1}^{+\infty} e^{-\frac{E}{k_{B}T}}}$
    \item E = n$E_{0}$ = $\frac{\Sigma_{n=1}^{+\infty} E_{0} \times e^{-\frac{E_{0}}{k_{B}T}}}{\Sigma_{n=1}^{+\infty} e^{-\frac{E_{0}}{k_{B}T}}}$
\end{itemize}
Conclusion : le modèle de Max Planck corrige la divergence de la densité spectrale u($\lambda$,T) pour l'UV (faible $\lambda$) (catastrophe ultraviolette)
\end{document}