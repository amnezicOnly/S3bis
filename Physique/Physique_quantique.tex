\documentclass{article}
%\usepackage{graphicx} % Required for inserting images
\usepackage{amssymb}
\usepackage{amsmath}
%\usepackage{amsfonts}
\usepackage{extarrows}
%\usepackage{soul}
\usepackage{enumitem}
\usepackage{varwidth}
\usepackage[T1]{fontenc}

\author{}
\date{}
\title{Physique quantique}

\begin{document}
\maketitle

La mécanique quantique, est un domaine de la physique apparu au XXe siècle, qui explique le comportement au niveau atomique et subatomique de la matière associée avec l'énergie, que la physique dite classique ne peut expliquer.\newline
\indent En 1905, Einstein utilise les résultats de Planck et explique que la lumière est composé de photons, impliquant une quantification de l'échange d'énergie \textbf{et} de l'énergie elle-même. C'est cette quantification qui rentre en contradiction avec la physique traditionnelle.

\section{Différence entre mécanique traditionnelle et quantique:}
\begin{itemize}
    \item{traditionnelle}
        \begin{itemize}
        \item à l'échelle macroscopique
        \item énergie continue
        \item position déterminée avec une bonne précision
        \end{itemize}
    \item{quantique}
        \begin{itemize}
        \item à l'échelle nanoscopique
        \item énergie discontinue (valeurs discrètes)
        \item notion de quanta: plus petite quantité d'énergie indivisible de valeur h$\nu$= E
        \[
            E = h\nu = \frac{hc}{\lambda}\Longrightarrow \nu=\frac{c}{\lambda}
            \quad
            \begin{varwidth}{\displaywidth}
                \begin{itemize}[nosep]
                    \item E: énergie (J)
                    \item h: constante de Planck (6,63$\times$10$^{-34}$J$\cdot$s$^{-1}$)
                    \item $\nu$ et $\lambda$: longueur d'ondes
                    \item c: célérité de la lumière de la vide (3$\times$10$^{8}$m$\cdot$s$^{-1}$)
                \end{itemize}
            \end{varwidth}
        \]
        \item notion de probabilité : la position de l'élément étudié est incertaine
        \item phénomènes interprétés grâce à la mécanique quantique (analyse spectrale qui permet d'identifier la matière en ayant analysé le spectre émis par celle-ci):
            \begin{itemize}
            \item photo-électrique (création d'un courant suite à l'interaction entre le photon et l'électron (libre) du métal)
            \item corps noir (TD1)
            \item modèle de Bohr (TD2)
            \end{itemize}
        \end{itemize}
\end{itemize}
\section{Corps noir}
Un corps noir est un objet physique théorique absorbant la totalité de l'énergie électro-magnétiques qu'il reçoit et la restitue entièrement sous forme d'un rayonnement thermique. En physique traditionnelle (loi de Rayleigh-Jeans), l'intensité des rayons renvoyés peut être donnée par la fonction suivante :
\[
    u(\lambda,T) = \frac{8\pi}{c\lambda^{2}}\langle E \rangle
\quad
\begin{varwidth}{\displaywidth}
    \begin{itemize}[nosep]
        \item u: densité spectrale (W$\cdot$m$^{3}$)
        \item T: température de rayonnement (K)
        \item c: célérité de la lumière dans le vide (3$\times$10$^{8}$m$\cdot$s${-1}$)
        \item $\lambda$: longueur d'ondes (nm)
        \item $\langle$E$\rangle$: énergie moyenne d'un oscillateur
    \end{itemize}
\end{varwidth}
\]

On peut calculer $\langle$E$\rangle$ via la formule suivante:
\[
    <E> = \frac{\int_{0}^{+\infty} Ee^{-\frac{E}{k_{B}T}}dE}{\int_{0}^{+\infty} e^{-\frac{E}{k_{B}T}}dE}
\quad
\begin{varwidth}{\displaywidth}
    \begin{itemize}[nosep]
        \item $k_{B}$: constante thermodynamique
        \item T : température du rayonnement
    \end{itemize}
\end{varwidth}
\]
Le numérateur représente l'intégrale de la densité de probabilité de l'énergie E, et le dénominateur permet de normaliser le résultat.\newline
Les rayonnements coulissent des UV(faibles longueurs d'ondes) aux IR(grandes longueurs d'ondes).\newline\newline
Le problème de la loi de Rayleigh-Jeans est qu'on arrive à la conclusion que l'intensité spectrale U (définie par U($\lambda$,T) = $\int_{0}^{+\infty} u(\lambda,T)$) tend vers +$\infty$ quand les longueurs tendent vers 0. C'est ce qu'on appelle la "catastrophe ultraviolette" car le résultat est absurde. On voit donc ici les limites de la physique traditionnelle.

\section{Proposition de Max Planck}
Le point important de la proposition de Max Planck est que l'énergie ne peut prendre que certaines valeurs discrètes, càd que l'énergie est discontinue.
\begin{itemize}
    \item Énergie discrète (E = n$E_{0}$) et non continue
    \item <E> = $\frac{\Sigma_{n=1}^{+\infty} E \times e^{-\frac{E}{k_{B}T}}}{\Sigma_{n=1}^{+\infty} e^{-\frac{E}{k_{B}T}}}$
    \item E = n$E_{0}$ = $\frac{\Sigma_{n=1}^{+\infty} E_{0} \times e^{-\frac{E_{0}}{k_{B}T}}}{\Sigma_{n=1}^{+\infty} e^{-\frac{E_{0}}{k_{B}T}}}$
\end{itemize}
Conclusion : le modèle de Max Planck corrige la divergence de la densité spectrale u($\lambda$,T) pour l'UV (faible $\lambda$) (catastrophe ultraviolette).
\end{document}