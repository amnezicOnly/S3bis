\documentclass{article}
\usepackage{graphicx} % Required for inserting images
\usepackage{amssymb}
\usepackage{amsmath}
%\usepackage{amsfonts}
\usepackage{extarrows}
%\usepackage{soul}
\usepackage{enumitem}
\usepackage{varwidth}
\usepackage[T1]{fontenc}

\author{}
\date{}
\title{Correction TD1 Physique quantique}

\begin{document}
\maketitle

\noindent\textbf{Exercice de cours : Catastrophe ultraviolette}\newline
\indent Le corps noir est l’objet théorique suivant : il absorbe parfaitement la totalité de l’énergie électromagnétique qu’il reçoit et, atteignant l’équilibre thermique à une température T, la restitue entièrement sous forme d’un rayonnement constitué d’un spectre de longueurs d’onde $\lambda$, rayonnées chacune à une intensité, donnée par la fonction u($\lambda$, T), appelée densité spectrale d’énergie.\newline
\indent La thermodynamique classique a proposé de décrire le rayonnement du corps noir en considérant les molécules comme de petits oscillateurs, rayonnant chacun une énergie E avec une certaine probabilité. La physique classique propose que l’énergie rayonnée E prend ses valeurs dans $[0;+\infty]$ ; des valeurs donc continues. On note <E> l’énergie moyenne d’un oscillateur.\newline\newline
\noindent\textbf{Partie 1 : Descritpion par la loi de Rayleigh-Jeans (Énergie continue)}\newline
Dans le cadre de la thermodynamique classique, la loi de Rayleigh-Jeans aboutit à l'expression suivante pour la densité spectrale d'énergie:
\[
    u(\lambda,T) = \frac{8\pi}{c\lambda^{2}}
\quad
\begin{varwidth}{\displaywidth}
    \begin{itemize}[nosep]
        \item u: densité spectrale (W$\cdot$m$^{3}$)
        \item c: célérité de la lumière dans le vide (3$\times$10$^{8}$m$\cdot$s${-1}$)
    \end{itemize}
\end{varwidth}
\]

\noindent
L'énergie moyenne <E> d'un oscillateur se calcule de la manière suivante:
\[
    <E> = \frac{\int_{0}^{+\infty} Ee^{-\frac{E}{k_{B}T}}dE}{\int_{0}^{+\infty} e^{-\frac{E}{k_{B}T}}dE}
\quad
\begin{varwidth}{\displaywidth}
    \begin{itemize}[nosep]
        \item $k_{B}$: constante thermodynamique
        \item T : température du rayonnement
    \end{itemize}
\end{varwidth}
\]

\noindent
Le numérateur représente l'intégrale de la densité de probabilité de l'énergie E, et le dénominateur permet de normaliser le résultat.\newpage
\begin{enumerate}
    \item Montrer que le terme <E> a pour valeur $k_{B}T$, en utilisant une intégration par parties. \newline \newline\newline
    Posons Nm et D tel que <E> = $\frac{Nm}{D}$\newline\newline
    D = $\int_{0}^{\infty} e^{-\frac{E}{k_{B}T}} dE$ = $[-k_{B}T\times e^{-\frac{E}{k_{B}T}}]_{0}^{\infty}$ = -$k_{B}$T(0-1) = $k_{B}$T \newline \newline
    Nm = $\int_{0}^{\infty} E\times e^{-\frac{E}{k_{B}T}}dE$, par intégration par parties, on obtient :\newline\newline
    Nm = $[-k_{B}T\times E\times e^{-\frac{E}{k_{B}T}}]_{0}^{\infty}$ - ($\int_{0}^{\infty} -k_{B}T\times e^{-\frac{E}{k_{B}T}}dE$)\newline\newline = $k_{B}T [-Ee^{-\frac{E}{k_{B}T}}]_{0}^{\infty} + k_{B}T[-k_{B}Te^{-\frac{E}{k_{B}T}}]_{0}^{\infty}$ \newline\newline= $k_{B}T[0-0] - (k_{B}T)^{2} [e^{-\frac{E}{k_{B}T}}]_{0}^{\infty}$\newline\newline = $-(k_{B}T)^{2} [e^{\alpha} - e^{0}] = -(k_{B}T)^{2} (0-1)$ = $(k_{B}T)^{2}$ \newline\newline

    Donc <E> = $\frac{Nm}{D}$ = $\frac{(k_{B}T)^{2}}{k_{B}T}$ = $k_{B}$T\newline\newline

    L'énergie totale du rayonnement émis par le corps noir est donnée par l'intégration de la densité d'énergie sur toutes les longueurs d'ondes:
    \[ U(\lambda,T) = \int_{0}^{+\infty} u(\lambda,T)d\lambda\]

    U($\lambda$,T) est l'intensité rayonnée par le corps noir (en W$\cdot$m$^{2}$)
    \item Expliquer à partir de cette intégrale pourquoi la loi de Rayleigh-Jeans a marqué un tournant nommé "Catastrophe ultraviolette".\newline\newline\newline
    $\int_{0}^{+\infty} u(\lambda,T)$ = $\int_{0}^{\alpha} \frac{8\pi}{c\lambda^{2}}\times k_{B}T d\lambda$\newline\newline
    = $\frac{8\pi}{c}\times k_{B}T \times \int_{0}^{\alpha} \frac{1}{\lambda^{2}}d\lambda$ = $\frac{8\pi}{c}\times k_{B}T \times [\frac{-1}{\lambda}]_{0}^{\alpha}$\newline\newline
    $\Longrightarrow$ $\lambda \rightarrow 0 \Longrightarrow$ U($\lambda$,T) diverge.
\end{enumerate}

\newpage
\textbf{Partie 2: Descritpion par la loi de Planck(Énergie discontinue)}\newline
Nous allons maintenant nous intéresser à la loi de Planck, qui a proposé de corriger la loi de Rayleigh-Jeans en introduisant la quantification de l’énergie du système corps noir : le terme <E> est alors calculé en supposant que l’énergie E d’un oscillateur ne peut prendre qu’un nombre discret de valeurs, multiples d’une énergie $E_{0}$.
\begin{enumerate}
    \setcounter{enumi}{2}
    \item Réécrire la relation donnant <E> dans ces conditions.
    \item Calculer ce terme, en utilisant un résultat sur les suites géométriques au numérateur, puis en remarquant un lien de dérivation entre le dénominateur et le numérateur.
    \item On obtient donc la densité spectrale d'énergie suivante :
\[ u(\lambda,T) = \frac{8\pi}{c\lambda^{2}}<E> = \frac{8\pi}{c\lambda^{2}} \times \frac{E_{0}}{e^{\frac{E_{0}}{k_{B}T}}-1} \] avec $E_{0}$ un quanta (plus petite quantité indivisible d'énergie)\newline
Pour corriger la loi de R-J, il faut corriger la divergence u($\lambda$,T) pour les petites longueurs d'ondes.\newline
Justifier la proposition de Planck: $E_{0} = \frac{hc}{\lambda}$ avec h la constante de Planck.\newline
\begin{figure}[h]
    \centering
    \includegraphics[scale=0.7]{catastrophe_ultraviolette.png}
    \caption{Spectre du rayonnement du corps noir (intensité du rayonnement en fonction de la longueur d’onde) pour des températures données (couleurs : modèle de Planck ; noir : modèle de Rayleigh-Jeans)}
\end{figure}

\indent\textit{Remarques:}\textit{Avec la loi de Planck, l’on trouve une énergie totale rayonnée U(T)$\alpha$T$^{4}$ , ce qui est en accord avec les observations, et revient à la loi de Stefan-Boltzman, antérieure à la catastrophe ultraviolette.}
\end{enumerate}


\end{document}