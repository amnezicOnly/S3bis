\documentclass{article}
\usepackage[a4paper, total={6in, 8in}]{geometry}
\usepackage{graphicx} % Required for inserting images
\usepackage{amssymb}
\usepackage{amsmath}
\usepackage{amsfonts}
\usepackage{extarrows}
\usepackage{soul}
\usepackage{enumitem}
\usepackage{varwidth}
\usepackage[T1]{fontenc}
\tolerance=1
\emergencystretch=\maxdimen
\hyphenpenalty=10000
\hbadness=10000

\author{}
\date{}
\title{Quantique TD3}

\begin{document}
\maketitle

\noindent\textbf{Ex1: Équation de Schrodinguer}\newline
L'équation de Schrodinguer est équivalente au PFD.
\[
    H\psi = E\psi
    \quad
    \begin{varwidth}{\displaywidth}
        \begin{itemize}[nosep]
            \item H = $-\frac{\hbar^{2}}{2m}\Delta + V$: opérateur Hamiltonien
            \item E: opérateur 
        \end{itemize}
    \end{varwidth}
\]

Pour rappel : $\Delta$ est l'opérateur Laplacien qui correspond à la somme des dérivés secondes partielles de chaque variables.\newline

$-\frac{\hbar^{2}}{2m}\Delta\psi + V\psi = E\psi$ qui donne $\psi$(x,y,z,t) = fonction d'onde qui caractérise l'état de la particule donc donne:\newline
$\left\{
    \begin{array}{l}
        \text{la probabilité de présence} \\
        \text{l'énergie de la particule}
    \end{array}    
$\newline\newline
$\left\{
    \begin{array}{l}
        \text{état stationnaire} \Longrightarrow \psi(x,y,z) \\
        \text{système à une dimension} \Longrightarrow \psi(x)
    \end{array}    
$
\newline
On en conclut donc que l'équation de Schrodinguer à l'état stationnaire et à une dimension :
\[
    -\frac{\hbar^{2}}{2m}\frac{d^{2}}{dx^{2}}\psi(x) + V\psi(x) = E\psi(x)    
\]
On note que c'est une dérivé totale car dérivé partielle mais il n'y a qu'une variable.\newline

La densité de probabilité de présence dP se calcule ainsi : dP = $|\psi (x)|^{2}$dx qui correspond au carré du module de la fonction d'onde $\psi(x)$.\newline

La répartition de la probabilité de présence dans une intervalle donné P se calcule ainsi : $\int_{a}^{b}|\psi(x)|^{2}dx$.\newline

La condition de la normalisation (probabilité de trouver la particule dans l'espace ]-$\infty$;+$\infty$[):\newline P = 1 = $\int_{-\infty}^{+\infty}|\psi(x)|^{2}dx$
\end{document}
