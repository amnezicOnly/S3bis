\documentclass{article}
\usepackage[a4paper, total={6in, 8in}]{geometry}  
%\usepackage{graphicx} % pour insérer des images

% pour utiliser des notations scientifiques
\usepackage{amssymb}
\usepackage{amsmath}
\usepackage{amsfonts}

\usepackage{extarrows} % pour afficher les flèches de logiques (implique, équivalent à ,etc...)
\usepackage{soul} % utilité à troiver
\let\oldemptyset\emptyset
\usepackage[T1]{fontenc}

\tolerance=1
\emergencystretch=\maxdimen
\hyphenpenalty=10000

\author{}
\date{}
\title{Résumé de cours}

\begin{document}
\maketitle
\section{Introduction à la physique quantique}
La mécanique quantique est un domaine de la physique apparu au XXe siècle, qui explique le comportement au niveau atomique et subatomique de la matière associée avec l'énergie que la physique dite classique ne peut expliquer.\newline
\indent En 1905, Einstein utilise les résultats de Planck et explique que la lumière est composé de photons, impliquant une quantification de l'échange d'énergie \textbf{et} de l'énergie elle-même. C'est cette quantification qui rentre en contradiction avec la physique traditionnelle.\newline\newline

\noindent Différence entre mécanique traditionnelle et quantique:
\begin{itemize}
    \item{traditionnelle}
        \begin{itemize}
        \item à l'échelle macroscopique
        \item énergie continue
        \item position déterminée avec une bonne précision
        \end{itemize}
    \item{quantique}
        \begin{itemize}
        \item à l'échelle nanoscopique
        \item énergie discontinue (valeurs discrètes)
        \item notion de quanta: plus petite quantité d'énergie indivisible de valeur h$\nu$= E
        \[
            E = h\nu = \frac{hc}{\lambda}\Longrightarrow \nu=\frac{c}{\lambda}
            \quad
            \begin{varwidth}{\displaywidth}
                \begin{itemize}[nosep]
                    \item E: énergie (J)
                    \item h: constante de Planck (6,63$\times$10$^{-34}$J$\cdot$s$^{-1}$)
                    \item $\nu$ et $\lambda$: longueur d'ondes
                    \item c: célérité de la lumière de la vide (3$\times$10$^{8}$m$\cdot$s$^{-1}$)
                \end{itemize}
            \end{varwidth}
        \]
        \item notion de probabilité : la position de l'élément étudié est incertaine
        \item phénomènes interprétés grâce à la mécanique quantique (analyse spectrale qui permet d'identifier la matière en ayant analysé le spectre émis par celle-ci):
            \begin{itemize}
            \item photo-électrique (création d'un courant suite à l'interaction entre le photon et l'électron (libre) du métal)
            \item corps noir (TD1)
            \item modèle de Bohr (TD2)
            \end{itemize}
        \end{itemize}
\end{itemize}

\section{Corps noir}
Un corps noir est un objet physique théorique absorbant la totalité de l'énergie électro-magnétique qu'il reçoit et la restitue entièrement sous forme d'un rayonnement thermique. En physique traditionnelle (loi de Rayleigh-Jeans), l'intensité des rayons renvoyés peut être donnée par la fonction suivante :
\[
    u(\lambda,T) = \frac{8\pi}{c\lambda^{2}}\langle E \rangle
    \quad
    \begin{varwidth}{\displaywidth}
        \begin{itemize}[nosep]
            \item u: densité spectrale (W$\cdot$m$^{3}$)
            \item T: température de rayonnement (K)
            \item c: célérité de la lumière dans le vide (3$\times$10$^{8}$m$\cdot$s$^{-1}$)
            \item $\lambda$: longueur d'ondes (nm)
            \item $\langle$E$\rangle$: énergie moyenne d'un oscillateur
        \end{itemize}
    \end{varwidth}
\]

On peut calculer $\langle$E$\rangle$ via la formule suivante:
\[
    \langle E \rangle = \frac{\int_{0}^{+\infty} Ee^{-\frac{E}{k_{B}T}}dE}{\int_{0}^{+\infty} e^{-\frac{E}{k_{B}T}}dE}
    \quad
    \begin{varwidth}{\displaywidth}
        \begin{itemize}[nosep]
            \item $k_{B}$: constante thermodynamique
            \item T : température du rayonnement
        \end{itemize}
    \end{varwidth}
\]

Le point important de la proposition de Max Planck est que l'énergie ne peut prendre que certaines valeurs discrètes, càd que l'énergie est discontinue.
\begin{itemize}
    \item Énergie discrète (E = n$E_{0}$) et non continue
    \item $\langle$E$\rangle$ = $\frac{\Sigma_{n=1}^{+\infty} nE_{0} \times e^{-\frac{nE_{0}}{k_{B}T}}}{\Sigma_{n=1}^{+\infty} e^{-\frac{nE_{0}}{k_{B}T}}}$
\end{itemize}
Le modèle de Planck corrige la divergence théorique de la densité rayonnée dans le domaine des UV (faibles longueurs d'ondes) et montre que le corps noir rayonne une énergie totale finie.

\section{Modèle de Bohr}
Le concept de quantum d'énergie $E_{0} = h\nu = \frac{hc}{\lambda}$ s'avère utile pour expliquer l'effet photo-électrique, sur des modèles d'une taille équivalente ou inférieure à celle de molécules ou d'atomes. On parlera à ces échelles d'apparition d'\textit{effets quantiques}.\newline
C'est l'énergie $E_{i}$ (énergie d'irradiaton) qui justifie la présence de seuils pour permettre l'effet photoélectrique
\begin{figure}[h]
    \centering
    \includegraphics[scale=0.5]{photoelectrique.png}
    \caption{Effet photoélectrique expliqué par les quanta $h\nu$}
\end{figure}

\begin{enumerate}
    \item Le modèle de l'atome est alors le modèle dit \textit{planétaire}, où l'on considère que les électrons sont présents sur des orbites et gravitent autour du noyau. Bohr postule que l'électron gravite sur des orbites circulaires stables successives (des couches), sans rayonner d'énergie car autrement il s'"effondrerait" sur le noyau.
    \item L'électron ne change d'orbite qu'en absorbant ou en rayonnant une certaine quantité d'énergie.
    \item (Conséquence de (1)) Le moment cinétique de l'électron sur une orbite est constant. Dans ce cours, nous dirons simplement que pour que l'orbite soit stable, elle doit vérifier:
    \[
        mrv = n\frac{h}{2\pi} = n\hbar
        \quad
        \begin{varwidth}{\displaywidth}
            \begin{itemize}[nosep]
                \item m: masse de l'électron (kg)
                \item v: vitesse de l'électron (m$\cdot$s$^{-1}$)
                \item r: le rayon de l'orbite circulaire (m)
                \item h: la constante de Planck (6,62$\times$ 10$^{-34}$ m$^{2}\cdot$kg$\cdot$s$^{-1}$)
                \item n: numéro de la couche électronique (n$\in\mathbb{N^{*}}$)
            \end{itemize}
        \end{varwidth}
    \]
\end{enumerate}

\noindent On a aussi ces deux relations:
\begin{itemize}
    \item \textbf{Rayons des orbites successives:} $r_{n} = a_{0}n^{2}$ avec $a_{0}$ le rayon de Bohr
    \item \textbf{Énergie de l'électron sur l'orbitale n:} $E_{n} = \frac{E_{1}}{n^{2}}, n\in\mathbb{N^{*}}$; $E_{1}$ l'énergie du niveau fondamental 
\end{itemize}
L'énergie des électrons et le rayon des orbitales successives sont quantifiées.
\begin{figure}[h]
    \centering
    \includegraphics[scale=0.5]{figure_2.png}
    \caption{Quantification des niveaux d'énergie / orbites dans l'atome d'hydrogène}
\end{figure}

\begin{itemize}
    \item L'énergie d'ionisation $E_{i}$ correspond en fait à la différence entre les énergies de l'état ionisé et de l'état fondamental $E_{i} = E_{final} - E_{initial} = 0 - E_{1} = -E_{1}$
    \item Un électron passe d'une couche n à m en absorbant(excitation)/cédant(désexcitation) de l'énergie (un quanta + précisément). L'échange d'énergie se calcule via la formule suivante :
    \[ \Delta E_{n\to m} = |E_{final(m)}-E_{initial(n)}| = h_{m,n}\nu = \frac{hc}{\lambda_{m,n}}  \Longrightarrow \Delta E augmente \rightarrow \lambda diminue \]
\end{itemize}

\section{Principe d'incertitude d'Heisenberg}
L'hypothèse de De Broglie va mener à la notion de "fonction d'onde" d'une particule. De la physique quantique va émerger le \textit{principe d'incertitude d'Heisenberg}:
\[
    \Delta x \Delta p \geqslant \frac{\hbar}{2}
    \quad
    \begin{varwidth}{\displaywidth}
        \begin{itemize}[nosep]
            \item x:
            \item p: quantité de mouvement
            \item $\hbar = \frac{h}{2\pi}$ avec h la constante de Planck (6,62$\times$ 10$^{-34}$ m$^{2}\cdot$kg$\cdot$s$^{-1}$)
        \end{itemize}
    \end{varwidth}
\]
$\Longrightarrow \Delta p = \Delta (mv) = m\Delta v \Longrightarrow m\Delta x\Delta v \geqslant \frac{\hbar}{2}$\newline\newline
On a du mal à calculer la vitesse et la position en même temps.\newline
Cette formule découle de la dualité onde-corpuscule (l'électron se comporte comme une particule \textbf{et} une onde).\newline
$\left\{
    \begin{array}[l]
        p_{e^{-}} = m_{e^{-}}v_{e^{-}} = \frac{h}{\lambda_{e^{-}}} \\
        \lambda_{e^{-}} = \frac{h}{m_{e^{-}}v_{e^{-}}} = \lambda_{B}
    \end{array}    
$

\section{Formules et infos utiles}
\subsection{Corps noir}
\begin{itemize}
    \item densité spectrale u($\lambda$,T):
    \[
        u(\lambda,T) = \frac{8\pi}{c\lambda^{2}}\langle E \rangle
        \quad
        \begin{varwidth}{\displaywidth}
            \begin{itemize}[nosep]
                \item u: densité spectrale (W$\cdot$m$^{3}$)
                \item T: température de rayonnement (K)
                \item c: célérité de la lumière dans le vide (3$\times$10$^{8}$m$\cdot$s$^{-1}$)
                \item $\lambda$: longueur d'ondes (nm)
                \item $\langle$E$\rangle$: énergie moyenne d'un oscillateur
            \end{itemize}
        \end{varwidth}
    \]
    \item Énergie moyenne $\langle$E$\rangle$ d'un oscillateur (formule classique):
    \[
        \langle E \rangle = \frac{\int_{0}^{+\infty} Ee^{-\frac{E}{k_{B}T}}dE}{\int_{0}^{+\infty} e^{-\frac{E}{k_{B}T}}dE}
        \quad
        \begin{varwidth}{\displaywidth}
            \begin{itemize}[nosep]
                \item $k_{B}$: constante thermodynamique
                \item T : température du rayonnement
            \end{itemize}
        \end{varwidth}
    \]
    Cette formule a pour conséquence la catastrophe ultraviolette.
    \item Énergie moyenne $\langle$E$\rangle$ d'un oscillateur (avec correction de Max Planck):
    \[
        \langle E \rangle = \frac{\Sigma_{n=1}^{+\infty} nE_{0}e^{-\frac{nE_{0}}{k_{B}T}}dE}{\Sigma_{n=1}^{+\infty} e^{-\frac{nE_{0}}{k_{B}T}}dE} = \frac{8\pi}{c\lambda^{2}}\times\frac{E_{0}}{e^{\frac{E_{0}}{k_{B}T}}-1}
        \quad
        \begin{varwidth}{\displaywidth}
            \begin{itemize}[nosep]
                \item $k_{B}$: constante thermodynamique
                \item T : température du rayonnement
                \item $E_{0}$ : énergie indivisible ($E_{0}=\frac{hc}{\lambda}$)
            \end{itemize}
        \end{varwidth}
    \]
\end{itemize}
\subsection{Modèle de Bohr}
\begin{itemize}
    \item $r_{n} = n^{2}a_{0}$
    \item $E_{n} = \frac{E_{1}}{n^{2}}$ avec $E_{1}$= -13,6eV\newline
    Rappel : 1ev = $1,6\times 10^{-19}$J
    \item $\Delta E_{n\to m} = |E_{m}-E_{n}| = h\nu = \frac{hc}{\lambda}$ 
    \item
    \begin{figure}[h]
        \centering
        \includegraphics[scale=0.5]{abs_emi.png}
        \caption{Absorption et émission d'un électron}
    \end{figure}
    \item $\left\{
        \begin{array}[l]
            \text{désexcitation : émission d'un photon} \\
            \text{excitation : absorption d'un photon}
        \end{array}    
    $
    \item $E_{tot} = E_{c}+E_{p_{\text{électrique}}} = \frac{1}{2}mv^{2} - e\frac{ke}{r}$
\end{itemize}


\end{document}
