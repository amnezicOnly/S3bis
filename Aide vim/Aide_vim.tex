\documentclass{article}
\usepackage[a4paper, total={6in, 8in}]{geometry}  
%\usepackage{graphicx} % pour insérer des images

% pour utiliser des notations scientifiques
\usepackage{amssymb}
\usepackage{amsmath}
\usepackage{amsfonts}

\usepackage{extarrows} % pour afficher les flèches de logiques (implique, équivalent à ,etc...)
\usepackage{soul} % utilité à troiver
\let\oldemptyset\emptyset
\usepackage[T1]{fontenc}

\author{Amnézic}
\date{}
\title{Aide Vim}

\begin{document}
\maketitle
\newpage
\tableofcontents
\newpage
\section{Introduction}
Pour avoir une aide en cas de problème : :help <sujet>.





\section{Actions classiques}
On prendra n le nombre d'éléments que l'on utilisera. On notera que dans la plupart des cas, si n=1, alors on a pas besoin de le mettre (on peut faire la commande sans le n). "." répète la dernière action réalisé sauf "u","ctr-r" et les commandes commençant par ":".








\subsection{Se déplacer}
On peut se déplacer de colonne en colonne et ligne en ligne avec les flèches directionnelles mais dans certaines commandes et documentations, les flèches seront utilisées comme ça
\begin{itemize}
    \item $\uparrow$ : k
    \item $\rightarrow$ : l
    \item $\downarrow$ : j
    \item $\leftarrow$ : h
\end{itemize}
La force de Vim réside dans ces commandes qui permettent d'aller beaucoup plus vite:
\begin{itemize}
    \item \textit{n}x : se déplace de n caractères
    \item \textit{n}w :(word) se déplace de n mots
    \item \textit{n}b :(begin) se déplace au début du n-ième mot avant le curseur (incluant le mot pointé)
    \item \textit{n}e :(end) se déplace à la fin du n-ième mot après le curseur (incluant le mot pointé)
    \item \textit{n}ge :(end) se déplace à la fin du n-ième mot avant le curseur (incluant le mot pointé)
    \item 0 : permet de se déplacer au début de la ligne courante
    \item $\$$ : permet de se déplacer à la fin de la ligne courante
    \item :\textit{n} : permet de se déplacer à la n-ième ligne
\end{itemize}





\subsection{Sélectionner}
Ici, nous utiliserons beaucoup le mode visuel (visual mode) ce qui va nous permettre de voir (par surlignage) ce que nous sélectionnons. Les commandes qui vont suivre seront toutes implicitement précédées par V:
\begin{itemize}
    \item el(n fois)<Enter> : surligne les n premiers caractères depuis le curseur et les supprime
    \item j(n fois) : 
\end{itemize}





\subsection{Supprimer}
\begin{itemize}
    \item d\textit{n}x : supprime les n premiers caractères depuis le curseur
    \item d\textit{n}w : supprime les n premiers mots depuis le curseur
    \item d\textit{n}e : supprime les n premiers mots depuis le curseur, sauf le dernier caractère
    \item d$\$$ : supprime le reste de la ligne
    \item dd : supprime toute la ligne courante
    \item \textit{n}dd : supprime les n lignes à partir du curseur (sur et en dessous)
    \item 
\end{itemize}
\subsection{Modifier}
\begin{itemize}
    \item c\textit{n}w[mot à placer]<Esc> : remplace les n mots depuis le curseur par [mot à placer] (à revoir)
    \item c$\$$ : supprime toute la ligne et passe en insert mode
    \item \textit{n}r[char] : remplace les n premiers caractères sur lesquel est le curseur par [char] (len(char) toujours égale à 1)
    \item J : à n'importe quel endroit de la ligne, permet de supprimer le prochain retour à la ligne
\end{itemize}





\subsection{Ajouter}
\begin{itemize}
    \item O : permet de rajouter une ligne au dessus du curseur et de s'y rendre
\end{itemize}





\subsection{Recherche}
On posera \textit{x} un caractère unique
\begin{itemize}
    \item \textit{n}f\textit{x} : se rend sur la n-ième occurrence de \textit{x} après le curseur
    \item \textit{n}F\textit{x} : se rend sur la n-ième occurrence de \textit{x} avant le curseur
\end{itemize}




\subsection{Action sur le fichier}
\begin{itemize}
    \item gg : aller au début du fichier 
    \item G : aller en bas du fichier
    \item :w : sauvegarde le fichier
    \item :q! : quitte le fichier sans sauvegarder\newline
    $\Longrightarrow$ ZZ est éqivalent à :wq
    \item pour ouvrir plusieurs fichiers en même temps:
    \begin{itemize}
    	\item les fichiers "les uns derrière les autres":
    	\begin{enumerate}
    		\item vim file1 file2 ... fileN
		\item :first : pour aller au premier fichier
		\item :next : pour aller au fichier suivant
		\item :previous : pour aller au fichier précédent
		\item :last : pour aller au dernier fichier
    	\end{enumerate}
	\item les fichiers côte à côte:
	\begin{enumerate}
		\item vim -o(horizontal) ou -O(vertical) file1 ... fileN
		\item Ctrl-W [direction] : se déplace dans la fenêtre indiquée par la direction
		\item 
	\end{enumerate}
    \item zo/zc : ouvre/ferme une scoop
    \item 
    \end{itemize}

\end{itemize}
\end
