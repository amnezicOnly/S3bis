\documentclass{article}
\usepackage{graphicx} % Required for inserting images
\usepackage{amssymb}
\usepackage{amsmath}
\usepackage{amsfonts}
\usepackage{extarrows}
\usepackage{soul}
\tolerance=1
\emergencystretch=\maxdimen
\hyphenpenalty=10000
\hbadness=10000
\let\oldemptyset\emptyset
\usepackage[T1]{fontenc}

\author{}
\date{}
\title{Midterm Algo 2027}

\begin{document}
\maketitle
\newpage
\newpage

\section{Questions de cours}
\begin{enumerate}
    \item Donnez une méthode de hachage direct:
    \item Le problème qui apparait avec le hachage coalescent:
    \item Une collision primaire est:
    \item L'ordre d'un graphe non-orienté est:
    \item Un sommet isolé est:
    \item Le tableau des demi-degrés extérieurs des sommets de G:
    \begin{center}
    \begin{tabular}{ |c| c| c| c| c| c| c| c| c| }
        \hline
        1 & 2 & 3 & 4 & 5 & 6 & 7 & 8 & 9 \\
        \hline
          &   &   &   &   &   &   &   &   \\
        \hline
    \end{tabular}
    \end{center}
    \item Le graphe est-il fortement connexe: Oui        Non
    \item Si non, combien possède t-il de composantes fortement connexes?
    \item S'ils existent, les sommets de G de degré différent de 4 sont:
    \item S'ils existent, les sommets de G de demi-degré intérieur égal à 0 sont:
\end{enumerate}

\section{Arbres généraux}
\section{B-arbres}


\end{document}