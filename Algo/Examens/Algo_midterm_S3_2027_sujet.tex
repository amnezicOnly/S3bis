\documentclass{article}
\usepackage{graphicx} % Required for inserting images
\usepackage{verbatim}
\usepackage[a4paper, total={6in, 8in}]{geometry}
%\usepackage{soul}
%\usepackage[T1]{fontenc}

\author{}
\date{}
\title{Midterm Algo 2027}

\begin{document}
\maketitle

\noindent\textbf{Exercice 1 (Haches et graphes... - \textit{6 points})}
\begin{enumerate}
    \item Donnez une méthode de hachage direct.
    \item Quel problème apparait avec le hachage coalescent?
    \item Qu'est-ce qu'une collision primaire?
    \item Qu'est-ce que l'ordre d'un graphe non-orienté?
    \item Un sommet isolé est:\newline\newline
    Soit le graphe G=<S,A> orienté tel que : S={1,2,3,4,5,6,7,8,9} et\newline
    A={(1,2),(1,6),(2,3),(2,5),(3,1),(3,4),(3,5),(4,5),(4,8),(6,2),(6,5),(7,5),(7,6),(7,8),(7,8),(8,5),(8,9),(9,4),(9,7)}
    \item Remplir le tableau des demi-degrés extérieurs des sommets G.
    \item Le graphe est-il fortement connexe?
    \item Si non, combien possède t-il de composantes fortement connexes?
    \item S'ils existent, les sommets de G de degré différent de 4 sont:
    \item S'ils existent, les sommets de G de demi-degré intérieur égal à 0 sont:
\end{enumerate}

\noindent\textbf{Exercice 2 (Level from x - \textit{5 points})}
Insérer figure 1 ici\newline
Écrire la fonction  \verb|keys_after(T:tree,x:int)| qui retourne la liste des clés dans l'arbre général T (implémentation "classique") se trouvant après la \textbf{première} avleur x trouvée sur le même niveau que x.\newline

\noindent\textbf{Exercice 3 (Average subtrees - \textit{6 points})}\newline
\indent Dans un arbre général, on cherche à savoir si un noeud n'a aucun sous-arbre dont la moyenne des clés est strictement supérieur à la clé de ce noeud.\newline
\indent Par exemple, l'arbre de la figure 2 ne respecte pas la propriété: la moyenne des clés du premier sous-arbre du noued contenant 4 est supérieur à 4.\newline
Insérer figure 2 et 3\newline
Par contre l'arbre de la figure 3 respecte bien la propriété :
\begin{itemize}
  \item Le noeud contenant 15:
  \begin{itemize}
    \item la moyenne des clés du premier sous-arbre est 2
    \item la moyenne des clés du deuxième sous-arbre est 5 = (8+11+2+5+0+4)/6
  \end{itemize}
  \item Noeud contenant 8: la moyenne des clés du premier sous-arbre est 5, inférieure à 8
  \item Tous les autres sous-arbres sont réduits à une feuille dont la clé est inférieure à celle du père.
\end{itemize}

Écrire la fonction \verb|average_subtrees(B:TreeAsBin)| qui vérifie si l'arbre B en implémentation premier fils-frère droit respecte la propriété.\newline

\noindent\textbf{Exercice 4 (B-arbre: insertions et suppressions - \textit{3 points})}\newline
Pour chaque question, utiliser le principe "à la descente" (principe de précaution) vu en td (hors bonus).
\begin{enumerate}
  \item Dessiner l'arbre après insertions successives des valeurs 21, 42 et 8 dans l'arbre de la figure 4.\newline
  Insérer figure 4
  \item Dessiner l'arbre après suppression de la valeur 80 dans l'arbre de la figure 5.\newline
  Insérer figure 5
\end{enumerate}

\newpage
\noindent\textbf{Annexes}
\begin{itemize}
  \item Arbres généraux
  \begin{itemize}
    \item Implémentation classique
    \begin{itemize}
      \item T : classe Tree
      \item T.key
      \item T.children: listes des fils ([] pour les feuilles)
      \item T.nbchildren = len(T.children)
    \end{itemize}
    \item Implémentation premier fils-frère droit
    \begin{itemize}
      \item B : classe TreeAsBin
      \item B.key
      \item B.child: le premier fils
      \item B.sibling: le frère droit
    \end{itemize}
  \end{itemize}
  \item Listes : comme d'habitude : len, range, min, max, abs, append
  \item Les files (les méthodes de la classe Queue, que l'on supposse importée)
  \begin{itemize}
    \item Queue(): retourne une nouvele file
    \item q.enqueue(x): enfile l'élément x dans q
    \item q.dequeue(): supprime et renvoie le premier élément de q
    \item q.isempty(): teste si q est vide
  \end{itemize}
\end{itemize}

\textbf{Vos fonctions}\newline
\indent Vous pouvez également écrire vos propres fonctions à condition qu'elles soient documentées: donnez leurs des spécifiacations (on doit savoir ce qu'elles font).\newline
\indent Dans tous les cas, la dernière fonction doit être celle qui répond à la question.

\end{document}
